\documentclass{article}
\usepackage{graphicx}
\usepackage[utf8]{inputenc}
\usepackage[italian]{babel}

\graphicspath{{./images/}}

\makeindex
\begin{document}

\begin{titlepage}
    \centering
    \vspace*{\fill}
    \Huge\textbf{Progetto finale di \\ Reti Logiche}\\
    \vspace{5mm} %5mm vertical space
    \Large Prof. Gianluca Palermo - Anno di corso 2020-21\\
    \vspace{5mm} %5mm vertical space
    \large Francesco Pastore - Codice persona: 10629332\\
    \vspace{100mm}
    \includegraphics[scale=0.7]{logo.png}
    \vspace*{\fill}
\end{titlepage}

\tableofcontents
\pagebreak

\section{Introduzione}
\noindent
Il metodo di equalizzazione dell’istogramma di una immagine è un metodo pensato per
ricalibrare il contrasto di una immagine quando l’intervallo dei valori di intensità sono molto
vicini effettuandone una distribuzione su tutto l’intervallo di intensità, al fine di incrementare il
contrasto. \\
Lo scopo del progetto è di implementare una variante semplificata di quest'algoritmo tramite un componente hardware descritto in VHDL.

\section{Implementazione}

\subsection{Segnali utilizzati}
\subsection{Stati}

\begin{itemize}
    \item \textbf{RESET:} One entry in the list
    \item \textbf{READ\_NUM\_COLS\_REQ:} One entry in the list
    \item \textbf{READ\_NUM\_COLS:} One entry in the list
    \item \textbf{READ\_NUM\_ROWS\_REQ:} One entry in the list
    \item \textbf{READ\_NUM\_ROWS:} One entry in the list
    \item \textbf{READ\_PIXELS\_START:} One entry in the list
    \item \textbf{READ\_NEXT\_PIXEL\_REQ:} One entry in the list
    \item \textbf{READ\_NEXT\_PIXEL:} One entry in the list
    \item \textbf{CHECK\_MIN\_MAX:} One entry in the list
    \item \textbf{WRITE\_START:} One entry in the list
    \item \textbf{EQUALIZE\_PIXEL:} One entry in the list
    \item \textbf{WRITE\_NEW\_PIXEL:} One entry in the list
    \item \textbf{DONE:} One entry in the list
\end{itemize}

\section{Test}
\subsection{Casi di test principali}
\subsection{Risultati dei test di post sintesi}

\end{document}